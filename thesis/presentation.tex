\documentclass[12pt]{article}


\usepackage{amsmath}
\usepackage{bm}
\usepackage{float}
\usepackage{amssymb}
\usepackage{graphicx}
\usepackage{bigints}
\usepackage{mathrsfs}
\usepackage{dcolumn}
\usepackage{mathtools}
\usepackage{tikz}
\usepackage{esint}
\usepackage{physics}
\usepackage[shortlabels]{enumitem}
\usepackage{parskip}
\usepackage[hidelinks]{hyperref}
\usepackage{dsfont}

%\newcommand{\rank}{\text{rank}}
%\newcommand{\real}{\text{Re}}
\newcommand{\imag}{\text{Im}}
\newcommand{\tens}{\otimes}


\usepackage[utf8]{inputenc}
\usepackage{breqn}
\usepackage[superscript,biblabel]{cite}
\usepackage{siunitx}



\title{Presentation}
\author{Quinlan Arnold}
\date{13/1/2022}

\begin{document}
Hamiltonian:
\begin{equation}
    H=H_0+\sum_k u_k H_k
\end{equation}
\section{Cost function}
\subsection{Fidelity cost}
The base cost function employed in GRAPE is $J=1-\Phi$ where $\Phi\in [0,1]$ is  the fidelity of a target unitary $U_t$ with the unitary resulting from the control fields, $U_f$. The fidelity and its derivative are calculated as\cite{rowland_implementing_2012}
\begin{align}
    \Phi&=\bra{U_t}\ket{U_f},\\
    \frac{\partial\Phi}{\partial u_{kj}} &= -2\text{Re}\bra{P_j}\ket{i\delta t\mathcal{H}_k X_j}\bra{X_j}\ket{P_j}
\end{align}
using the standard inner product for matrices
\begin{equation}
    \bra{A}\ket{B} = \frac{1}{N}\text{tr}(A^\dagger B)
\end{equation}
giving base cost function 
\begin{align}
    J &= 1-\bra{U_t}\ket{U_f},\\
    \nabla_{jk}J &= 2\text{Re}\bra{P_j}\ket{i\delta t\mathcal{H}_k X_j}\bra{X_j}\ket{P_j}
\end{align}z

\subsection{Fluctuation cost}
Term added to cost function to reduce fluctuation in control field amplitudes.
\begin{align}
    J_f &= \alpha\sum_{j=0}^{N-1}\sum_{k=1}^m \left[u_k(j+1)-u_k(j)\right]^2,\\
    \nabla_{jk} J_f&= 2\alpha\left[2u_k(j) - u_k(j-1) - u_k(j+1)\right],\quad 1\leq j\leq N-1,\\
    \nabla_{0k}J_f &= \nabla_{Nk} = 0.
\end{align}

\section{Implementation}
In plots shown below, two control fields $\pi/2$ out of phase with each other are applied which are resonant with the desired transition.

\section{Single qubit gates}
\begin{figure}[H]
    \centering
    \includegraphics{presPlots/X gate.pdf}
    \caption{X gate}
    \label{fig:my_label}
\end{figure}
\begin{figure}[H]
    \centering
    \includegraphics{presPlots/H gate.pdf}
    \caption{Y gate}
    \label{fig:my_label}
\end{figure}

\begin{figure}[H]
    \centering
    \includegraphics{presPlots/Y gate.pdf}
    \caption{Hadamard gate}
    \label{fig:my_label}
\end{figure}

\section{2 qubit CNOTSs}

Free evolution Hamiltonian matrix:
\begin{align}
    H_0 = \begin{pmatrix}
    E_0+A_1\hbar+A_2\hbar+J &0 &0 &0\\
    0 &A_1\hbar-A_2\hbar-J &2J &0\\ 
    0 &2J &-A_1\hbar+A_2\hbar-J &0 \\ 
    0 &0 &0 &-E_0-A_1\hbar-A_2\hbar+J
    \end{pmatrix}
\end{align}



\begin{figure}[H]
    \centering
    \includegraphics{presPlots/CNOT_bigA2.pdf}
    \caption{CNOT using control Hamiltonian at resonance frequency $\omega_{34}=(E_3-E_4)/\hbar$. Hyperfine value for qubit 1 is $A_1=10^{13}$ Hz, unrealistically large.}
    \label{fig:my_label}
\end{figure}

\begin{figure}[H]
    \centering
    \includegraphics{presPlots/CNOT_bigA1_10ns.pdf}
    \caption{CNOT for 10ns, still working well.}
    \label{fig:my_label}
\end{figure}

\begin{figure}[H]
    \centering
    \includegraphics{presPlots/CNOT_realisticA1.pdf}
    \caption{CNOT with more realistc hyperfine $A_1=10^{10}$ Hz. Fidelity = 0.3, and amplitude oscillating very rapidly.}
    \label{fig:my_label}
\end{figure}


\section{Issues}
\begin{itemize}
    \item Can't get CNOT to work for realistic values of hyperfine coupling.
    \item CNOT has phase error which seems to depend on the duration of the gate.
    \item Currently not making any approximations, so the rapid precession due to static 2T field requires many timesteps to simulate accurately.
\end{itemize}

\end{document}